\chapter{Basics}
\emph{%
This chapter
}
%
Spatial information entities (si-entities), for example Twitter tweets, Flickr image entries, or even Google queries, have the form of a 4-tuple with the following features: 
\begin{equation*}
si\text{-}entity = (location, keywords, time\text{-}stamp, sender),
\end{equation*}
where
\begin{description}\setlength{\itemsep}{-2mm}
\item[$location$] represents either the exact position based on GPS, or an approximation via IP address or triangulation with available WLANs.
\item[$keywords$] are the text for the entity -- a search query, twitter message, user-generated tags, or something alike.
\item[$time$-$stamp$] is the date and time the entity has been sent\,/\,posted.
\item[$sender$] contains the user-name respectively -id of the entity source.
\end{description}
Using social network functionality, a fifth feature could be added, $receiver$, containing the same form of user identification, but for the target of the entity.

Based on these features, similar si-entities can be grouped\,/\,clustered together. Similarity is usually defined as a distance function\,/\,metric -- the euclidean distance is a popular one. A metric $d(x,y)$ has to be non-negative ($d(x, y) \ge 0$), symmetric ($d(x, y) = d(y, x)$) and to follow the triangle inequality ($d(x, z) \le d(x, y) + d(y, z)$). Clustering is then just the search for elements close to each other. Of course there are other forms of clustering algorithms.
%
The distribution of the entities to the clusters is directly depended of the used distance function (and therefore the features) and the applied algorithm.

\bigskip
\noindent
The focus of this thesis will be the comparisons between the distributions of the
\begin{itemize}\setlength{\itemsep}{-2mm}
\item $location$ feature $d_{loc}(x,y)$, the 
\item $keywords$ feature $d_{key}(x,y)$, and a 
\item combination of both $d_{comb}(x,y) = \alpha \cdot d_{loc}(x,y) + \beta \cdot d_{key}(x,y)$ ($\alpha \in (0,1), \beta = 1-\alpha$).
\end{itemize}

These features will be clustered based on a highly widespread clustering algorithm: DBSCAN, which is a density-based method.
The relationship between those cluster distributions will be analyzed based on various cluster measures (for example \emph{perplexity}). The influence of $\alpha$ and $\beta$ on the combined feature is also of interest.

A graph-based clustering algorithm based on \cite{Zhou2009} will be implemented and adapted to the data set, in order to compare the resulting distributions with the previous base-work. A graph consists of nodes (entities) and edges representing a relationship. The $si-entities$ can be transformed to such a graph, enabling the comparison of these two different clustering techniques.

